\chapter{Background}\label{chap:background}

Sample citation: \cite{backens2018zhcalculus}

\section{The ZX-calculus}\label{sec:zx-background}
\ROUGH{In \cite{CD1}\cite{CD2}, the ZX-calculus was introduced, forming the
first of a family of diagrammatic languages for reasoning about quantum logic.
It is still the most accessible and well-explored of these languages, especially
with the availability of \cite{CKbook} , a comprehensive textbook on the subject.
Therefore, before going into the details of any particular language, it helps to
be familiar with the ZX-calculus.}

\TODO{Add reference to Penrose graphical notation, with proper cite}
\begin{TODOLIST}
Introduce here, in order:
\begin{enumerate}
\item Generators (incl. phase states)
\item Generators (incl. phase states)
\item Only Topology Matters
\item The spider laws (incl. phase spiders)
\item Strong complementarity (mention "homomorphism" and "bialgebra")
\item Weak complementarity as result of strong and OTM
\item The axioms (table)
\item The laws (table)
\end{enumerate}
\end{TODOLIST}
