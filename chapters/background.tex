\chapter{Background}\label{chap:background}


\section{The ZX-calculus}\label{sec:zx-background}
\ROUGH{In \cite{CD1}\cite{CD2}, the ZX-calculus was introduced, forming the
first of a family of diagrammatic languages (in the sense of Penrose's graphical
notation) for reasoning about quantum logic.  It is still the most accessible
and well-explored of these languages, especially with the availability of
\cite{CKbook} , a comprehensive textbook on the subject.  Therefore, before
going into the details of any particular language, it helps to be familiar with
the ZX-calculus.}

\TODO{Add reference to Penrose graphical notation, with proper cite}

\ROUGH{The ZX-calculus and its derivatives are all based around the same set of
principles: linear maps $\mathbb{C}^{(2^n)} \rightarrow \mathbb{C}^{(2^m)}$ (which corresponds to maps from $n$ to $m$ qubits) are
represented by graphs with $n$ dangling "input" wires on one side and $m$
dangling "output" wires on the other. In this thesis, we will follow the same
convention as in \cite{CKbook} by having data flow upwards, with input wires at
the bottom and output wires at the top: 
$$\intf{\tikzfig{example-nin-mout}} := f : \mathbb{C}^{(2^n)} \rightarrow
\mathbb{C}^{(2^m)}$$ 
where $\intf{\cdot}$ is the semantic evaluation bracket interpreting diagrams as
linear maps in the conventional way. It is important to note that alternative
semantics are also possible, from whose compatibility with the axioms of a particular
calculus various proofs regarding (in)completeness or minimality of those axioms
w.r.t. the standard interpretation can be derived. In this chapter and
\autoref{chap:phasefree} we will introduce the standard interpretations for the
various calculi, and in \autoref{chap:minimality} and
\autoref{chap:completeness} we will use such alternative interpretations to show
minimality and completeness.

All calculi share some semantic composition rules. $\otimes$-composition and
$\circ$-composition are represented as parallel resp. sequential composition:
$$\intf{\tikzfig{g-otimes-f}} := g \otimes f , 
  \intf{\tikzfig{g-circ-f}}   := g \circ f    $$

All calculi allow substitution: if two graphs are equal in a calculus, then
any two graphs differing only by which of the two they contain as a subgraph are
also equal:
$$\tikzfig{generic-f} = \tikzfig{generic-g} \Rightarrow \tikzfig{f-clouded} =
\tikzfig{g-clouded}$$

And while the following two rules are not universal in general, they are very
common (and correspond to desirable properties of the underlying system) and do
hold in all the calculi we'll consider: 
\begin{itemize}
    \item An empty wire of any length corresponds to the identity mapping: %
        ${\intf{\tikzfig{id-wire}}=id :\mathbb{C}^2 \rightarrow \mathbb{C}^2}$
    \item Only Topology Matters: if two graphs can be deformed into each other,
        without breaking connections or reordering inputs and outputs, they are
        the same graph:$$ \tikzfig{otm1} = \tikzfig{otm2} = \tikzfig{otm3} \neq
        \tikzfig{otm-counter}$$
\end{itemize}


 }


\begin{TODOLIST}
Introduce here, in order:
\begin{enumerate}
\item Generators (incl. phase states)
\item Customary interpretation of the generators.
\item Only Topology Matters
\item The spider laws (incl. phase spiders)
\item Strong complementarity (mention "homomorphism" and "bialgebra")
\item Weak/Hopf complementarity as result of strong and OTM
\item The axioms (table)
\item The laws (table)
\item Graph states? As application and for backref next section.
\end{enumerate}
\end{TODOLIST}

\section{The ZH-calculus}\label{sec:zh-background}

Sample citation: \cite{backens2018zhcalculus}
\begin{TODOLIST}
Introduce here, (in order?):
\begin{enumerate}
\item (cite)
\item Generators with their intended interpretation.
\item OTM, spider laws, bialg\&Hopf as in ZX
\item The axioms (table)
\item Toffoli
\item Hypergraph states? Original motivation, but not prominent here.
\end{enumerate} 
\end{TODOLIST}

\section{The ZX$\Delta$-calculus}\label{sec:zxd-background}

\begin{TODOLIST}
\begin{enumerate}
\item As ZX, but add triangle (cite)
\item Triangle and its interpretation.
\item The axioms (table)
\end{enumerate}
\end{TODOLIST}

\NOTE{Should ZW also be introduced? Not directly necessary, but very welcome
background and might be useful to be able to refer to as motivation for some
steps. Also, forms backbone of ZX$\Delta$ completeness.}

\section{!-box notation}\label{sec:bangbox}

\begin{TODOLIST}
\begin{enumerate}
\item The !-box, motivation, origin (cite), and semantics
\item Restate laws (which?) !-boxed.
\item !-box induction (cite), preferably with example.
\end{enumerate}
\end{TODOLIST}
