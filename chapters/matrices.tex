\chapter{Matrix form}\label{chap:matrices}
The original ZH-calculus completeness proof relied on rewriting to a normal form
which essentially represented the underlying matrix directly. While we can't
repeat the full result here for lack of the parametrised axioms it's based on,
we can define a matrix form, represent some numbers and some matrices, and
rewrite some diagrams to their corresponding matrices. This allows us to
construct equality proofs for some diagrams which have proven otherwise
intractable. Note that this purely syntactic sugar; everything defined in this
chapter is defined in terms of the phasefree ZH-calculus, and all proofs can be
translated back to yield valid though potentially lengthy proofs.

\section{Numbers}
\ROUGH{
First, we'll define a few formal "numbers" in our calculus (we can define more
later, but these will illustrate our setup nicely). Because our numbers are going
to be used as entries in a matrix, rather than as direct scalars, they'll need
to have at least one edge wire to connect to for manipulation. In fact, we'll
give them all exactly one output wire, and then define how to turn that into any
number of input and output wires. 
}

\begin{definition}
    A \textbf{number state} is a diagram
    $$\tikzfig{numstate} \text{ satisfying } \tikzfig{numstatedef}$$
    
    A \textbf{number} is a diagram (up to diagrammatic equality)
    $$\tikzfig{numdef}$$
    where $\tikzfig{numstate}$ is a number state.
\end{definition}
\begin{remark*}
    The diagrammatic equality provision, besides keeping our meaning of equality
    consistent, also means that all number states are, in fact, numbers.
\end{remark*}

Since we can cheat a bit by choosing diagrams that in the full ZH-calculus are
provably equal to the numbers we want, we'll do just that. Since this isn't the
full ZH-calculus and we can't use its rules to handle numbers, we'll limit
ourselves to a few we can prove specific cases of the missing rules for:
$$\tikzfig{numdefs}$$
Two things to note here: first, all these numbers are boldfaced and subscripted
`H'. This is to distinguish \NOTE{should this be distinguish or differentiate?}
them from the complex numbers which parametrize the full ZH-calculus. Second,
while the last rule is general, combining it with the first gives us
$$\tikzfig{minusone}$$
And for the higher arity versions, we define:
$$\tikzfig{aritydef}$$

\section{Matrices}\label{sec:matrices}
Recall that the original ZH-calculus in \cite{backens2018zhcalculus} used
indexing maps to define normal forms.  Since those maps did not use parameters
anywhere, we can use them here as well.  For completeness' sake, we repeat the
definition:
$$\tikzfig{bitmaskdef}$$
Where $\tikzfig{nottofirst}$ and $\tikzfig{nottozeroth}$ . We'll also want the
binary representation function 
$b^n : \mathbb{N} \rightarrow \mathbb{B}^n, i \mapsto \left( \left \lfloor
\frac{i}{2^{n-1}}\right \rfloor \text{mod } i \right) \ldots \left( \left \lfloor
\frac{i}{2^{0}}\right \rfloor \text{mod } i \right)$ . Where $n$ is obvious or
irrelevant, we'll leave it implicit, $b^n =: b$.

Now that we have both numbers and indices, we can define matrices within the
phasefree ZH-calculus:
$$\tikzfig{matrixdef}$$

Since this produces $O(n^2)$ nodes and $O(n^3)$ wires for maps $\mathbb{C}^n
\rightarrow \mathbb{C}^n$, this clearly doesn't produce very tractable diagrams
when we need to expand a matrix into its definition. Worse, some of the more
useful tricks cause a lot of $\hnum{1}$ entries in intermediate steps next to
the entries we're actually interested in (this is because $\hnum{1}$ is the
pointwise multiplicative identity, allowing us to construct diagrams by taking
Schur products of simpler diagrams). To mitigate this at least slightly (and in
specific cases we're interested in, significantly), we introduce the
\textbf{reduced normal form}


\section{Trinary matrices}
\begin{TODOLIST}
    Write meta-lemmata as already used informally.
    \begin{enumerate}
        \item Some numbers can be assigned a formal representation
        \item (reduced) normal forms can be written as a matrix of formally %
            represented numbers
        \item Schur products of normal forms can be computed pointwise,
            including in their matrix representation (straight from ZH2018)
        \item The generalised ortho rule/splitting lemma lets us do the
            topological part of contraction.
        \item For some matrices, including those only containing \{-1, 0, 1\}, we
            can prove the cases of the (A) rule needed to do the computational
            part of contraction
        \item We can de-introduce a factor $\frac{1}{2}$
    \end{enumerate}
\end{TODOLIST}

\section{Proving the (\&) rule}

\subsection{Pre-matrix steps}
First, we need to bring the diagrams into a shape that our tools can deal with,
since sequential composition isn't directly supported. Since the upper leg of
the left-hand side is not in reduced normal form, we'll first find that so we
can replace it in the diagram.\\
$\tikzfig{down-triangle-inv-to-rednorm}$
Now we can prepare the left-hand side:\\
$\tikzfig{and-lhs}$ 
$= \tikzfig{and-lhs-seq-rednorm}$ 
$= \tikzfig{and-lhs-star-rednorm}$
And doing the same for the right-hand side:
$\tikzfig{and-rhs}$
$= \tikzfig{and-rhs-star-rednorm}$\\

\subsection{Calculating the matrix}
Using our earlier theorems, we can now write the above equations in matrix
form:\\
$\tikzfig{and-lhs-small-matrices}$
$= $
