\chapter{Matrix form}\label{chap:matrices}
The original ZH-calculus completeness proof relied on rewriting to a normal form
which essentially represented the underlying matrix directly. While we can't
repeat the full result here for lack of the parametrised axioms it's based on,
we can define a matrix form, represent some numbers and some matrices, and
rewrite some diagrams to their corresponding matrices. This allows us to
construct equality proofs for some diagrams which have proven otherwise
intractable. Note that this purely syntactic sugar; everything defined in this
chapter is defined in terms of the phasefree ZH-calculus, and all proofs can be
translated back to yield valid though potentially lengthy proofs.

\section{Numbers}
\ROUGH{
First, we'll define a few formal "numbers" in our calculus (we can define more
later, but these will illustrate our setup nicely). Because our numbers are going
to be used as entries in a matrix, rather than as direct scalars, they'll need
to have at least one edge wire to connect to for manipulation. In fact, we'll
give them all exactly one output wire, and then define how to turn that into any
number of input and output wires. 
}

\begin{definition}
    A \textbf{number state} is a diagram
    $$\tikzfig{numstate} \text{ satisfying } \tikzfig{numstatedef}$$
    A \textbf{number} is a diagram 
    $$\tikzfig{numdef}$$
    up to diagrammatic equality, where $\tikzfig{numstate}$ is a number state.
\end{definition}
\begin{remark*}
    The diagrammatic equality provision, besides keeping our meaning of equality
    consistent, also means that all number states are, in fact, numbers.
\end{remark*}

Informally, we want to choose our number states such that plugging
$\tikzfig{notmeasure}$ into them yields a scalar whose standard interpretation
equals the complex number we've named our number state for. That is, we want
$$\forall \hnum{a} \intf{\tikzfig{num-a-scalar}} = a$$
Some examples of diagrams satisfying these constraints are
$$\tikzfig{numdefs}$$
It's easy to check in the full ZH-calculus that these equal the parametrized
H-boxes they replace. The rule for \hnum{-1} is given explicitly as a
simplifying exception to the rule for \hnum{-a}.

\subsection{Canonical sets of number states}\label{sec:canonsets}
Since normal forms depend for their uniqueness on uniquely representing each
number, we run into an issue not found in the full ZH-calculus: choosing a
representative for each interpretation-equivalence class of number states. Since
we can't provide an exhaustive list of all representable numbers, and the
preferred representation might depend on the techniques we want to apply, we
need to define a general criterion to recognize valid normal form components.
\begin{definition}
    A \textbf{canonical set} is a set $A$ of number states such that $\forall a,
    b \in A, \intf{a} = \intf{b} \Rightarrow a = b$ where $=$ is graph equality,
    rather than rewrite equality.
\end{definition}
\begin{remark}
    Any subset of a canonical set is itself canonical, since removing elements
    can never create a pair which violates the uniqueness condition.
\end{remark}

Now we are free to choose canonical sets based on convenience, and add numbers
as we need them, so long as we don't break the uniqueness condition in doing so.
Normal forms are well-defined so long as we never compare two normal forms based
on canonical sets whose union is not a canonical set.

\begin{lemma}\label{lem:canoncanon}
    The set $\left\{\tikzfig{numstate-a}\ \middle|\ \hnum{a} \in \{\hnum{-2}, \hnum{-1}, \hnum{-\frac{1}{2}}, \hnum{0},
    \hnum{\frac{1}{2}}, \hnum{1}, \hnum{2}\}\right\}$ is canonical
\end{lemma}
\begin{proof}
    Straightforward calculation shows that, for each \hnum{a},
    $\intf{\tikzfig{numstate-a}} =
    \begin{pmatrix}
        1 \\
        a
    \end{pmatrix}$, so every state has a different interpretation.
\end{proof}




\section{Matrices}\label{sec:matrices}
Recall that the original ZH-calculus in \cite{backens2018zhcalculus} used
indexing maps to define normal forms.  Since those maps did not use parameters
anywhere, we can use them here as well.  For completeness' sake, we repeat the
definition:
$$\tikzfig{bitmaskdef}$$
Where $\tikzfig{nottofirst}$ and $\tikzfig{nottozeroth}$ . We'll also want the
binary representation function 
$b^n : \mathbb{N} \rightarrow \mathbb{B}^n, i \mapsto \left( \left \lfloor
\frac{i}{2^{n-1}}\right \rfloor \text{mod } i \right) \ldots \left( \left \lfloor
\frac{i}{2^{0}}\right \rfloor \text{mod } i \right)$ . Where $n$ is obvious or
irrelevant, we'll leave it implicit, $b^n =: b$.

Now that we have both numbers and indices, we can define matrices within the
phasefree ZH-calculus:
\begin{definition}\label{def:matrix-form}
    A diagram is said to be in \textbf{matrix form} (and is called a
    \textbf{matrix diagram}) if it matches
    $$\tikzfig{matrix-M-diag-expanded}$$
    And any such diagram may be written in shorthand as 
    $$\tikzfig{matrix-diag-def}$$

    Furthermore, a diagram is said to be in \textbf{normal form} if it is in
    matrix form, has no input wires, and the number states form a canonical set.
    This matches the shape of normal forms in \cite{backens2018zhcalculus}.
\end{definition}

Since this produces $O(n^2)$ nodes and $O(n^3)$ wires for maps $\mathbb{C}^n
\rightarrow \mathbb{C}^n$, this clearly doesn't produce very tractable diagrams
when we need to expand a matrix into its definition. Worse, some of the more
useful tricks cause a lot of $\hnum{1}$ entries in intermediate steps next to
the entries we're actually interested in (this is because $\hnum{1}$ is the
pointwise multiplicative identity, allowing us to construct diagrams by taking
Schur products of simpler diagrams). To mitigate this at least slightly (and in
specific cases we're interested in, significantly), we make the following
observation: 
\begin{lemma}
    Given $n,m \in \mathbb{N}, B \subset {1 \ldots n} \times {1 \ldots m}, A =
    {1 \ldots n} \times {1 \ldots m}\\B$, we have 
    $$\tikzfig{rednormal-rule}$$
\end{lemma}
\begin{proof}
    $$\tikzfig{rednormal-proof}$$
\end{proof}

Based on this, we add the following definition
\begin{definition}
    A diagram is said to be \textbf{reduced matrix form} if it's equal to a
    matrix form with all elements corresponding to \hnum{1} removed. More
    generally, any variation on matrix forms can be \textbf{reduced} by removing
    all elements corresponding to \hnum{1}.
\end{definition}

\section{Calculation}\label{sec:calculation}
Given the notation above, we can add some limited arithmetic and basic
manipulations. Specifically, we'll define the product and the average on
numbers, prove some results mirroring extension, contraction, and convolution
from \cite{backens2018zhcalculus} and precompute the product and average for the
most useful cases.

\begin{definition}
    If \hnum{a} and \hnum{b} are defined, then \hnum{a \star b} is defined as
    $$\tikzfig{num-prod-def}$$
\end{definition}

\begin{definition}
    If \hnum{a} and \hnum{b} are defined, then \hnum{AVG(a, b)} is defined as
    $$\tikzfig{num-avg-def}$$
\end{definition}


\section{Trinary matrices}

\begin{definition}
    A matrix diagram \tikzfig{indexed-matrix-M} is in \textbf{trinary matrix
    form} and called a \textbf{trinary matrix diagram} (resp. \textbf{quintary
    matrix form} and \textbf{quintary matrix diagram}) if $\forall i,j \in
    \underline{n} \times \underline{m}, M_{i,j} \in \left\{\hnum{-1},
    \hnum{0},\hnum{1}\right\}$ (resp. $M_{i,j} \in \left\{\hnum{-2},\hnum{-1},
    \hnum{0},\hnum{1},\hnum{2}\right\}$). A (reduced) trinary or quintary matrix
    diagram without inputs is in (reduced) trinary resp. quintary normal form.
\end{definition}
\begin{remark}
    Since all the number states allowed by this definition are in the canonical
    set from \autoref{lem:canoncanon}, every trinary or quintary matrix diagram
    without inputs is in fact in normal form.
\end{remark}

\begin{lemma}\label{lem:extclosed}
    Given a diagram \tikzfig{normal-S-diag} in trinary normal form, we can bring
    the extended diagram \tikzfig{normal-S-extend} into trinary normal form.
\end{lemma}
\begin{proof}
    The proof mirrors that in \cite{backens2018zhcalculus}, but since that proof
    relies on the introduction axiom, which we don't have here, we need to show
    we can perform the same step for our numbers \hnum{-1}, \hnum{0}, and
    \hnum{1}
    $$\tikzfig{intro-minone-proof}$$
    $$\tikzfig{intro-zero-proof}$$
    $$\tikzfig{intro-one-proof}$$
    And now the rest of the proof is a straightforward translation from the
    original:
    $$\tikzfig{ext-closed-proof}$$
\end{proof}
\begin{corollary}
    Since bending and swapping wires merely rearranges index bits without
    destroying the matrix form, we can generalize the above to all matrices and
    arbitrary positions of the extension by transforming to and from normal
    form. So, for any trinary matrix $M$ and $i, j \geq 0$, there are trinary
    matrices $M'$ and $M''$ with
    $$\tikzfig{matrix-M-extend} = \tikzfig{matrix-Mprime} \text{ and }%
    \tikzfig{matrix-M-extend-down} = \tikzfig{matrix-Mdoubleprime}$$
    In practical use, the rule for determining the resulting matrix is that
    extending on output duplicates rows, extending on input duplicates columns,
    and duplication happens in chunks of $2^j$. This follows directly from the
    values of indices with either a $0$ or $1$ inserted after bit $i$ and the
    roles of the upper and lower index maps.
\end{corollary}

\begin{lemma}\label{lem:convclosed}
    Given two diagrams \tikzfig{normal-S1-diag},
    \tikzfig{normal-S2-diag} in
    trinary normal form, we can bring the convolution
    \tikzfig{normal-S1-S2-conv} into trinary normal form.
\end{lemma}
\begin{proof}
    Again, we follow \cite{backens2018zhcalculus}, after going through the
    necessary replacements of the multiplication axiom.
    $$\tikzfig{mult-minmin}$$
    $$\tikzfig{mult-zeroany}$$
    $$\tikzfig{mult-oneany}$$
    And now the rest of the proof is a straightforward translation from the
    original:
    $$\tikzfig{conv-closed-proof}$$
\end{proof}
\begin{corollary}
    Again, since bending wires doesn't destroy the property of being a matrix
    form, we can generalize to matrices: given trinary matrices
    $M\scriptstyle1$ and $M\scriptstyle2$
    with matching numbers of input and output wires, we can convolve their
    diagrams and get the resulting diagram into a trinary matrix form. The
    result is a diagram whose matrix is the Schur product of $M\scriptstyle1$
    and $M\scriptstyle2$: $(M{\scriptstyle1} \star  M{\scriptstyle2})_{i,j} =%
    M{\scriptstyle1}_{i,j} \star M{\scriptstyle2}_{i,j}$.
\end{corollary}

\begin{lemma}\label{lem:tricontrquint}
    Given a diagram \tikzfig{normal-S-diag} in trinary normal form, the
    contraction \tikzfig{normal-S-contract} can be brought into quintary normal
    form.
\end{lemma}
\begin{proof}
    Translating this proof from \cite{backens2018zhcalculus} is slightly more
    work, because we need to show the essence of the disconnect lemma (B.3)
    holds using our numbers, calculate the equivalent of the average axiom, and
    show that we're left with exactly one scalar spider left over. Then, because
    we lack an elegant proof for introduction on \hnum{2}, we instead have to
    apply the reverse directions of convolution and introduction to first
    isolate and then eliminate a factor of \hnum{\frac{1}{2}} from each
    component.


    The replication of the disconnect lemma is effectively one-to-one, though we
    will track scalars. Luckily, most of the heavy lifting is done by the
    generalized ortho lemma, so we only need to redo the final steps here.  By
    induction on $n$ (where the case $n=0$ is trivial using (ZS2)):
    $$\tikzfig{b3-claim} \text{(IH)}$$
    $$\tikzfig{b3-translated}$$

    Applying this, and lemma (3.3) from \cite{backens2018zhcalculus} to our
    left-hand diagram, we replicate most of the original proof:
    $$\tikzfig{contract-proof-replicate}$$
    But run into the lack of a general average rule. However, we can calculate
    the averages we'll encounter on trinary normal forms. Since we have a
    \tikzfig{half} everywhere, we'll add it to our left-hand side.
    $$\tikzfig{avg-minmin}$$
    $$\tikzfig{avg-onemin}$$
    $$\tikzfig{avg-zeromin}$$
    $$\tikzfig{avg-oneone}$$
    $$\tikzfig{avg-zeroone}$$
    $$\tikzfig{avg-zerozero}$$

    Finally, we need to show how each of the above outcomes can be written as
    a product with \hnum{frac{1}{2}}
    $$\tikzfig{intro-two-proof}$$
    $$\tikzfig{twice-min}$$
    $$\tikzfig{twice-minhalf}$$
    $$\tikzfig{twice-zero}$$
    $$\tikzfig{twice-half}$$
    $$\tikzfig{twice-one}$$

    Which allows us to take the final few steps, with the meaning of
    parenthesized numbers matching normal arithmetic:
    $$\tikzfig{contract-proof-finale}$$

\end{proof}
\begin{corollary}
    As before, given trinary matrix $M$, we can contract along any output or
    input and bring the result into quintary matrix form.
\end{corollary}

\begin{lemma}\label{lem:uniquematrices}
    Matrices consisting of canonical number states are unique: Given two matrix
    diagrams $\tikzfig{matrix-M1-diag},\tikzfig{matrix-M2-diag}$ whose
    constituent number states belong to the same canonical set, we have
    $$\intf{\tikzfig{matrix-M1-diag}} = \intf{\tikzfig{matrix-M2-diag}} \iff
    \tikzfig{matrix-M1-diag} = \tikzfig{matrix-M2-diag}$$
\end{lemma}
\begin{proof}
    Since the interpretation of a matrix diagram is given by
    $\intf{\tikzfig{matrix-M-diag}}_{i,j} = e_i^T \intf{\tikzfig{matrix-M-diag}}
    e_j$, and basis vectors $e_i^T,e_j$ are represented by sequences of
    \tikzfig{greymeasure} and \tikzfig{notmeasure}, 
    resp.  
    \tikzfig{greystate} and \tikzfig{notstate} 
    we can extend our concept of index maps to index measurements and states by
    dropping the output resp. input wires,
    and take $\neg b$ to be the bitwise complement of $b \in \mathbb{B}^n$ to
    encode $e_i^T$ and $e_j$ as 
    $$e_i^T = \intf{\tikzfig{index-i-measure}} \text{ resp. }
    e_j = \intf{\tikzfig{index-j-state}}$$
    Applying these to our matrix, we get 
    $$e_i^T\intf{\tikzfig{matrix-M-diag}}e_j =
    \intf{\tikzfig{index-i-measure}} \circ
    \intf{\tikzfig{matrix-M-diag}} \circ
    \intf{\tikzfig{index-j-state}}$$
    $$= \intf{\tikzfig{matrix-M-selection-steps}}$$
    where $\oplus$ is the bitwise exclusive or. Now, where either the resulting
    index state or measurement is unequal to $11 ... 11$ (that is, if $i \not =
    k$ or $j \not = l$), we have at least one \tikzfig{greymeasure} or
    \tikzfig{greystate} connected to the number in that instance of the !-box.
    Since our numbers are invariant under swaps and leg flipping, we can bring
    it to the side as \tikzfig{greyside} and per 
    $$\tikzfig{number-destruction-proof}$$
    we have
    $$\tikzfig{index-selection-demo}$$
    Giving $\intf{\tikzfig{matrix-M-diag}} = 
    \begin{pmatrix}
        \intf{\numstate{M_{1,1}}} & \cdots & \intf{\numstate{M_{1,m}}} \\
        \vdots & \ddots & \vdots \\
        \intf{\numstate{M_{n,1}}} & \cdots & \intf{\numstate{M_{n,m}}}
    \end{pmatrix}$.

    Now since $\intf{\tikzfig{num-a}} = 
    \begin{pmatrix}
        \intf{\tikzfig{num-greymeasure}} \\
        \intf{\tikzfig{num-notmeasure}}
    \end{pmatrix}$ and $\intf{\tikzfig{num-greymeasure}} =
    \intf{\tikzfig{empty}} = 1$,
    states with differing interpretation can only differ in their second
    component $\intf{\tikzfig{num-notmeasure}}$.  This gives us
    $$
    \intf{\tikzfig{numstate-M1ij}} = \intf{\tikzfig{numstate-M2ij}}
    \iff
    \intf{\tikzfig{numstate-M1ij-not}} = \intf{\tikzfig{numstate-M2ij-not}}
    ,$$
    $$
    \intf{\tikzfig{matrix-M1-diag}} = \intf{\tikzfig{matrix-M2-diag}} 
    \iff
    \forall (i,j) \in \underline{n} \times \underline{m},
    \intf{\tikzfig{numstate-M1ij-not}} = \intf{\tikzfig{numstate-M2ij-not}}
    ,$$
    $$\text{and so }
    \intf{\tikzfig{matrix-M1-diag}} = \intf{\tikzfig{matrix-M2-diag}} \iff
    \forall (i,j) \in \underline{n} \times \underline{m},
    \intf{\tikzfig{numstate-M1ij}} = \intf{\tikzfig{numstate-M2ij}}
    $$
    \ROUGH{The core proof step is based on the above, the uniqueness of the
    number states, and the fact that a matrix is completely determined by the
    number state at each index}

    Because of how matrix diagrams are defined, they can only differ in what
    number states appear where. Therefore,
    $$
    {\tikzfig{matrix-M1-diag}} = {\tikzfig{matrix-M2-diag}} \iff
    \forall (i,j) \in \underline{n} \times \underline{m},
    \tikzfig{numstate-M1ij} = \tikzfig{numstate-M2ij}
    $$
    And since the number states form a canonical set, we have
    $$
    \intf{\tikzfig{numstate-M1ij}} = \intf{\tikzfig{numstate-M2ij}}
    \iff
    \tikzfig{numstate-M1ij} = \tikzfig{numstate-M2ij}$$
    Giving us 
    \begin{align*}
    \intf{\tikzfig{matrix-M1-diag}} = \intf{\tikzfig{matrix-M2-diag}} 
        &\iff
    \forall (i,j) \in \underline{n} \times \underline{m},
    \intf{\tikzfig{numstate-M1ij}} = \intf{\tikzfig{numstate-M2ij}} \\
    \forall (i,j) \in \underline{n} \times \underline{m},
    \intf{\tikzfig{numstate-M1ij}} = \intf{\tikzfig{numstate-M2ij}} 
        &\iff
    \forall (i,j) \in \underline{n} \times \underline{m},
    \tikzfig{numstate-M1ij} = \tikzfig{numstate-M2ij} \\
    \forall (i,j) \in \underline{n} \times \underline{m},
    \tikzfig{numstate-M1ij} = \tikzfig{numstate-M2ij} 
        &\iff
    \forall (i,j) \in \underline{n} \times \underline{m},
    \tikzfig{numstate-M1ij} = \tikzfig{numstate-M2ij} \\
    \forall (i,j) \in \underline{n} \times \underline{m},
    \tikzfig{numstate-M1ij} = \tikzfig{numstate-M2ij} 
        &\iff
    \tikzfig{matrix-M1-diag} = \tikzfig{matrix-M2-diag}
    \end{align*}
    Which completes the proof.


\end{proof}

Combining the above lemmata, we can take any number of diagrams in (reduced)
trinary matrix form, extend and convolve them as much as we want, contract them
once, and since the elements of a quintary matrix form a canonical set, if the
interpretation of the diagrams is the same, the matrices must be the same. Thus,
any two diagrams with the same interpretation can be rewritten into each other
so long as they both can be expressed as a contraction on some combination of
extensions and convolutions of trinary matrices. Note that this is a worst case;
some quintary matrices resulting from contraction happen to also be trinary, and
therefore can be extended, convolved, and contracted further.

\begin{theorem}
    Since $\intf{\tikzfig{h-toffoli}} = \intf{\tikzfig{triangle-toffoli}}$ and
    both sides can be expressed as a contraction on a trinary matrix,
    we have $\tikzfig{h-toffoli} = \tikzfig{triangle-toffoli}$ .
\end{theorem}
\begin{proof}
    \tikzfig{toffoli-decomp-parallel}
\end{proof}

\begin{TODOLIST}
    Write meta-lemmata as discussed
    \begin{enumerate}
        \item \sout{Extension on \{\hnum{-1},\hnum{0},\hnum{1}\} is closed.}
        \item \sout{Convolution on \{\hnum{-1},\hnum{0},\hnum{1}\} is closed.}
        \item \sout{Contraction on \{\hnum{-1},\hnum{0},\hnum{1}\} goes to %
            \{\hnum{-2},\hnum{-1},\hnum{0},\hnum{1},\hnum{2}\}.}
        \item \sout{Given unique number representations, matrices are unique.}
        \item \sout{\{\hnum{-2},\hnum{-1},\hnum{0},\hnum{1},\hnum{2}\} form %
            a set of unique number representations by the soundness of the %
            calculus.}
        \item \sout{The previous 3 statements give the uniqueness of diagrams w.r.t. %
            $\intf{-}$ after contraction on a trinary matrix}
        \item \sout{Since $\intf{\tikzfig{h-toffoli}} =
            \intf{\tikzfig{triangle-toffoli}} $, the above gives that
            \tikzfig{h-toffoli} = \tikzfig{triangle-toffoli} by turning both
            into quintary matrices: first extend and convolve to trinary
            matrices, then contract once to get a quintary matrix (which is in
            fact a trinary matrix, but that is a happy coincidence).}
    \end{enumerate}
\end{TODOLIST}

\begin{TODOLIST}
    Write meta-lemmata as already used informally.
    \begin{enumerate}
        \item Some numbers can be assigned a formal representation
        \item (reduced) normal forms can be written as a matrix of formally %
            represented numbers
        \item Schur products of normal forms can be computed pointwise,
            including in their matrix representation (straight from ZH2018)
        \item The generalised ortho rule/splitting lemma lets us do the
            topological part of contraction.
        \item For some matrices, including those only containing \{-1, 0, 1\}, we
            can prove the cases of the (A) rule needed to do the computational
            part of contraction
        \item We can de-introduce a factor $\frac{1}{2}$
    \end{enumerate}
\end{TODOLIST}

\section{Proving the (\&) rule}

\subsection{Pre-matrix steps}
First, we need to bring the diagrams into a shape that our tools can deal with,
since sequential composition isn't directly supported. Since the upper leg of
the left-hand side is not in reduced normal form, we'll first find that so we
can replace it in the diagram.\\
$\tikzfig{down-triangle-inv-to-rednorm}$
Now we can prepare the left-hand side:\\
$\tikzfig{and-lhs}$ 
$= \tikzfig{and-lhs-seq-rednorm}$ 
$= \tikzfig{and-lhs-star-rednorm}$
And doing the same for the right-hand side:
$\tikzfig{and-rhs}$
$= \tikzfig{and-rhs-star-rednorm}$\\

\subsection{Calculating the matrix}
Using our earlier theorems, we can now write the above equations in matrix
form:\\
$\tikzfig{and-lhs-small-matrices}$
$= $
