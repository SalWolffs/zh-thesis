\chapter{Matrix form}\label{chap:matrices}
The original ZH-calculus completeness proof relied on rewriting to a normal form
which essentially represented the underlying matrix directly. While we can't
repeat the full result here for lack of the parametrised axioms it's based on,
we can define a matrix form, represent some numbers and some matrices, and
rewrite some diagrams to their corresponding matrices. This allows us to
construct equality proofs for some diagrams which have proven otherwise
intractable. Note that this purely syntactic sugar; everything defined in this
chapter is defined in terms of the phasefree ZH-calculus, and all proofs can be
translated back to yield valid though potentially lengthy proofs.

\section{Numbers}
Before defining matrices, it's useful to consider what numbers we can
represent, and how. The number-representing diagrams we want for use in matrices
will need to be connected and maskable. To do this, we'll standardize numbers on
one output wire, and make them mimic H-box behaviour from the original paper:
plugging $\tikzfig{measure-grey}$ into them should make them disappear, and
plugging $\tikzfig{measure-not}$ into them should leave a scalar diagram which,
under the standard interpretation, translates to the number represented. We can
define the following numbers: \\
${\tikzfig{num-minus} = \hnum{-1}}$ , 
${\tikzfig{num-zero} = \tikzfig{grey-zero} = \hnum{0}}$ ,
${\tikzfig{num-half} = \hnum{\frac{1}{2}}}$ ,
${\tikzfig{num-one} = \hnum{1}}$ ,
${\tikzfig{num-neg-a} = \hnum{-a}}$

, so we will use diagrams with one output wire. We also
want them to be maskable, so we'll choose diagrams that disappear when plugged
with 


was proven complete by essentially embedding
arbitrary matrices into it as normal forms and proving every diagram could be
transformed into such a normal form. While we haven't been able to rewrite
arbitrary 



\section{Trinary matrices}
\begin{TODOLIST}
    Write meta-lemmata as already used informally.
    \begin{enumerate}
        \item Some numbers can be assigned a formal representation
        \item (reduced) normal forms can be written as a matrix of formally %
            represented numbers
        \item Schur products of normal forms can be computed pointwise,
            including in their matrix representation (straight from ZH2018)
        \item The generalised ortho rule/splitting lemma lets us do the
            topological part of contraction.
        \item For some matrices, including those only containing \{-1, 0, 1\}, we
            can prove the cases of the (A) rule needed to do the computational
            part of contraction
        \item We can de-introduce a factor $\frac{1}{2}$
    \end{enumerate}
\end{TODOLIST}

\section{Proving the (\&) rule}

\subsection{Pre-matrix steps}
First, we need to bring the diagrams into a shape that our tools can deal with,
since sequential composition isn't directly supported. Since the upper leg of
the left-hand side is not in reduced normal form, we'll first find that so we
can replace it in the diagram.\\
$\tikzfig{down-triangle-inv-to-rednorm}$
Now we can prepare the left-hand side:\\
$\tikzfig{and-lhs}$ 
$= \tikzfig{and-lhs-seq-rednorm}$ 
$= \tikzfig{and-lhs-star-rednorm}$
And doing the same for the right-hand side:
$\tikzfig{and-rhs}$
$= \tikzfig{and-rhs-star-rednorm}$\\

\subsection{Calculating the matrix}
Using our earlier theorems, we can now write the above equations in matrix
form:\\
$\tikzfig{and-lhs-small-matrices}$
$= $
