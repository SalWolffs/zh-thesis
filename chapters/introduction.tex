\chapter{Introduction}\label{chap:introduction}
\ROUGH{Graphical calculi are being investigated for their ability to represent
and simplify quantum circuits, widely used for reasoning about quantum
computing.}

\ROUGH{ZH is a relatively recent addition, shown to be complete, but using quite
powerful axioms, which aren't necessarily physical.}

\ROUGH{Research question: canwe limit the axioms of the ZH-calculus as described
to a phasefree version while preserving completeness?}

\ROUGH{Beside answering this main research question, it would also be nice to
achieve a more developed understanding of how to work with the ZH-calculus. The
original paper achieved completeness using a small set of fairly specific rules,
leaving us with a complete but largely unexplored calculus. To complement this,
we've made an attempt to find and include a somewhat comprehensive list of basic
and practical rewrite rules for working in the ZH-calculus.}

\section{Structure}\label{sec:thesisstruct}
\ROUGH{The structure of this thesis, then, will be as follows: }

\ROUGH{
In \autoref{chap:background}, we will provide an overview of past work in the
field, introducing first the ZX-calculus for context, then the ZH-calculus, and
finally the ZX$\Delta$-calculus which we will use for our completeness proof.
There will also be a primer on !-boxes, which we will make extensive use of in
later chapters.
}

\ROUGH{
In \autoref{chap:techniques}, we will go into the techniques we discovered for
the ZH-calculus. We'll provide a list of practical rewrite rules, and elaborate
on their use a bit. We'll show a few of the more interesting proofs, and leave
the rest to \autoref{chap:proofs}. 
}

\ROUGH{
Minimality of the axioms is considered in \autoref{chap:minimality}. Using a
combination of alternative models and invariants, we'll argue that the axioms we
have chosen are indeed minimal. 
}

\ROUGH{
Using the techniques from \autoref{chap:techniques}, we'll show the completeness
of our axioms in \autoref{chap:completeness}. This will be based on a
back-and-forth interpretation between the phasefree ZH-calculus and the
ZX$\Delta$-calculus.
}

\ROUGH{
Finally, in \autoref{chap:conclusion}, we'll recap our results, place them in
context, and consider future work.
}
