\chapter{Techniques}\label{chap:techniques}
\section{ZX analogues}
\ROUGH{The following rules will look familiar to those used to the ZX calculus.
They are true, useful, and easily proven. Since the proofs take up quite some
space and aren't particularly interesting, they are delegated to \autoref{chap:proofs}.}
\TODO{Add a table with all ZX-analogues laws here}

\section{The CZ-rule}
\begin{TODOLIST}
\begin{enumerate}
\item The rule
\item Its (brief) proof
\item Motivation (refer forward)
\end{enumerate}
\end{TODOLIST}

\section{Generalising the Ortho rule}
\ROUGH{Appendix B of \cite{backens2018zhcalculus} goes a long way towards
generalizing the (O) rule, but specialises it for their specific use case in
normal forms. We will repeat their proof here in shortened form, but add some
extra !-boxes and drop parameters to make it applicable to our case.
Additionally, we will add the relevant scalars. 

Having proven a more accessible form, we will then prove a different
generalisation before stating and proving a form which combines both. }

\section{Trinary matrices}
\begin{TODOLIST}
    Write meta-lemmata as already used informally.
    \begin{enumerate}
        \item Some numbers can be assigned a formal representation
        \item (reduced) normal forms can be written as a matrix of formally %
            represented numbers
        \item Schur products of normal forms can be computed pointwise,
            including in their matrix representation (straight from ZH2018)
        \item The generalised ortho rule/splitting lemma lets us do the
            topological part of contraction.
        \item For some matrices, including those only containing \{-1, 0, 1\}, we
            can prove the cases of the (A) rule needed to do the computational
            part of contraction
        \item We can de-introduce a factor $\frac{1}{2}$
    \end{enumerate}
\end{TODOLIST}

\section{Worked example: proving the idempotence rule}
\NOTE{This is the Z-H connection deduplication rule. Essentially, here goes the
last bit of Appendix D of (cite Wetering\&Wolffs), but with more elaboration.
CORRECTION: this can be done by trinary matrices. Either do the most elegant
one, or do both to demonstrate the idea.}

\section{Proving the (\&) rule}

\section{Working with $\Delta$}
\begin{TODOLIST}
\begin{enumerate}
\item Various shapes of $\Delta$ (note most common/useful one)
\item Rewrite rules involving $\Delta$
\item Proofs, or reference to \autoref{chap:proofs}
\end{enumerate}
\end{TODOLIST}



\section{The laws}
\ROUGH{Following the example of \cite[Ch.~10.3]{CKbook}, we conclude this chapter with a
complete list of all the laws we have found and consider useful. [FIXME: replace
with actual cases] have not been proven yet, their proofs can be found at the
end of \autoref{chap:proofs} . All the others have been mentioned and proven before. }

\begin{TODOLIST}
\begin{enumerate}
\item The axioms and definitions
\item Alternate forms of some axioms and definitions (e.g. (NOT), (\&),
($\Delta$))
\item Clearly marked repetition of all rules proven in
\cite{backens2018zhcalculus}
\item Rules from sections in this chapter
\item Schur decomposition?
\item Matrix representation of (reduced) normal forms.
\item Anything else
\end{enumerate}
\end{TODOLIST}
